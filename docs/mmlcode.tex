As far as possible, higher-order functions within the standard library are written with `foldr' as a
base. This provides a constant interface for all forms of list manipulation, and as such acts as the
higher-order `primitive' for microML\@. Figure~\ref{fig:foldr}

\begin{figure}[H]
    \begin{minted}[breaklines=true]{sml}
let foldr f acc xs = if empty? xs then acc else ((f (head xs)) (foldr f acc (tail xs)))
let map f xs = foldr (\x xs' -> (f x) : xs') [] xs
let filter p xs = foldr (\x y -> if (p x) then (x:y) else y) [] xs
let foldr1 f xs = foldr f (head xs) xs
    \end{minted}
    \caption{foldr in the standard library}
\label{fig:foldr}
\end{figure}

Logs, for reasons of efficiency, are primitives in the language, however other maths functions can
be written quite comfortably in microML.

\begin{figure}
    \begin{minted}[breaklines=true]{sml}
let sin x = 
    if x == 90 then 1 -- because of rounding errors, let's cheat a little here
    else if x == 180 then 0
    else let y = (x / 180) * pi 
          in y - (y^3 / 6) + (y^5 / 120) - (y^7 / 5070) + (y^9 / 362880) - (y^11 / 39916800) 
          -- hard code the factorials for efficiency

let cos x = 
    if x == 180 then (-1)
    else sin (90 - x)

let tan x = (sin x) / (cos x)
    \end{minted}
    \caption{Sine, cosine and tan in microML}
\label{fig:maths}
\end{figure}
