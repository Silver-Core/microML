\subsection{Operators}

MicroML has a full complement of operators. Table~\ref{table:operators}

\begin{table}
   % \resizebox{\textwidth}{!}
    \begin{tabu}{l r}
        Operator & Action \\
        \hline
        +   & addition \\
        -   & subtraction \\
        /   & division (produces float) \\
        *   & multiplication \\
        //  & integer division (produces integer) \\
        =   & assignment \\
        ==  & equality test \\
        $<$   & less than \\
        $<=$  & less than or equal \\
        $>$   & greater than \\
        $>=$  & greater than or equal \\
        $:$ & cons \\
        $++$ & joins lists or strings \\
        \textasciicircum & exponential \\
        \%  & integer modulo \\
        $>>$  & pipe operator \\
        \hline
        true & logical true \\
        false & logical false \\
        or  & logical or \\
        and & logical and \\
        not & logical not \\
        xor & logical xor \\
    \end{tabu}
    \caption{MicroML\@: arithmetical and logical operators}
\label{table:operators}
\end{table}

\subsection{Declarations}
Top-level declarations for variables and functions (simple and recursive) are prefaced with
\textit{let}:

\begin{minted}{sml}
    let x = 5
    let double x = x * 2
\end{minted}

\textit{Locally scoped declarations} are created with the let \dots in construction:

\begin{minted}{sml}
    let addHidden y = 
        let x = y * 2 - 3 
        in x + y
\end{minted}

The if -- then -- else construction is an \textit{expression} in microML, so there must be an
\textit{else}:

\begin{minted}{sml}
    if x == 0 then true else false
\end{minted}

MicroML supports \textit{anonymous functions} with a syntax similar to Haskell:

\begin{minted}{sml}
    let inc = \x -> x + 1
\end{minted}

These anonymous functions can also be used in pipes:

\begin{minted}{sml}
    microML>  (double 5) >> succ >> succ >> \x -> x^2
    microML> 144 : Number
    microML> (double 5) >> x -> x^2 >> succ >> succ
    microML> 102 : Number
\end{minted}

\subsection{Number Encodings}
MicroML supports encoding binary, octal and hex numbers:

\begin{minted}{sml}
    microML> 2#110
    microML> 6 : Number
    microML> 8#777
    microML> 511 : Number
    microML> 16#2bbad21
    microML> 45853985 : Number
    microML> 8#342 + 2#1111
    microML> 241 : Number
\end{minted}

\subsection{List Syntax}
Lists, being the fundamental data structure in microML, have a special syntax:

\begin{minted}{sml}
    microML> []   (* the empty list *)
    microML> [1,2,3] (* list of Number *)
\end{minted}

Ranges (from small to large only) can be created with the `to' syntax:

\begin{minted}{sml}
    microML> [1 to 5]
    microML> [1,2,3,4,5] : Number
    microML> [`a' to `e'] 
    microML> [`a', `b', `c', `d', `e'] : Char
\end{minted}

\subsection{Tuples}
Tuples are created with \{  and \}

\begin{minted}{sml}
    microML> {1, 2}
\end{minted}

\subsection{Indenting}
MicroML does not (yet) have an indentation sensitive parser. Declarations are bests written on one
line in a file, though the parser is usually sophisticated enough to disambiguate multiline function
declarations.
