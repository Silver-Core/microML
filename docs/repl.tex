The following commands are available in the repl (Table~\ref{tabel:repl}):

\begin{table}[ht]
    \begin{tabu}{l l r}
            Command & Arguments & Purpose \\
            \hline \\
            \textit{expr} & & Evaluate an expression \\
            :t & function & Check the type of a function \\
            :? :help & function / glossary & Read the help \\
            :clear & & Clear the terminal \\
            :using & lib name & Load a module from the standard library \\
            :q :quit & & Exit microML \\
            :load & filepath & Load a script into the repl \\
            :browse & & See the functions in the environment \\
            :pst & \textit{expr} & Pretty print parse tree to terminal \\
            :pstText & \textit{expr} & Print text form of parse tree \\
            :! & string & call the shell \\
    \end{tabu}
\caption{In-repl commands}
\label{tabel:repl}
\end{table}

Interactions in the repl are \textbf{line-orientated}. Multiline input is not yet supported.

The repl supports some configuration via \textit{microMLrc}, which is placed in the user's home
directory upon installation. The default file is see in Figure~\ref{fig:config}.

\begin{figure}
    \begin{minted}[breaklines=true]{bash}
## config file for microML

## simple ansi colour codes, see https://en.wikipedia.org/wiki/ANSI_escape_code for more details

[colourscheme]
bold = 1
number = 31
string = 32
char = 34
boolean = 33
error = 31
prompt = 33
    \end{minted}
    \caption{Configuration file for microML}
\label{fig:config}
\end{figure}
