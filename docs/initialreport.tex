\documentclass[11pt, a4paper]{article}
\usepackage[utf8]{inputenc}
\usepackage[english]{babel}
\usepackage{minted}
\usepackage{palatino}
\usepackage{geometry}
\usepackage[hidelinks]{hyperref}
\geometry{a4paper,left=15mm,right=15mm, top=1cm, bottom=2cm}
\usepackage{color, colortbl}
\definecolor{Seagreen}{rgb}{0.18, 0.54, 0.34}
\definecolor{Lawngreen}{rgb}{0.48, 0.99, 0}
\definecolor{LRed}{rgb}{1, 0.8, 0.8}
\definecolor{GoldenRod}{rgb}{0.93, 0.65, 0.12}
\usepackage{verbatimbox}

\begin{document}
\title{Functional Programming on the BBC micro:bit: A Proposal}
\author{David Kelly}

\maketitle

\section{Functional Programming and the micro:bit} The BBC micro:bit is a microprocessor device aimed at
bringing programming to young children and teenagers. At present it supports a number of 
different programming languages, most visibly \textit{MicroPython, JavaScript} and \textit{TouchDevelop}.
These languages, while mainstream and popular,\footnote{With the exception of \textit{TouchDevelop}
    of course.} all fundamentally support the same programming paradigm, the imperative. Doubtless it is
vital for all would-be coders to have knowledge and experience with the imperative/procedural
approach to structuring code, but it is not the only approach, and perhaps not the best for
beginners or those with only a passing interest in coding. It is the contention of the author that a
syntactically simple, (relatively) \textit{pure} functional language would make for an ideal 
teaching tool on the micro:bit, allowing the student to focus almost entirely on the problem domain, 
and a great deal less on syntax and \textit{boilerplate}.
Functional languages are increasingly being used in every area of real-world applications. Regarding 
robotics, in which Rae has expressed an interest, there has already been some research conducted 
in the use of the declarative style for low level robotics programming\footnote{Lambda in Motion: Controlling
    Robots with Haskell, Peterson John, Paul Hudak and Conal Elliott}. This at least demonstrates
that the functional style is applicable to the problem area, but further research would be required
to test the suitability of the micro:bit as a low-level controller. Certainly, anything which can be
programmed in microPython could also be written (possible more briefly and with fewer bugs) in a
functional language.\footnote{Rae has also brought to my attention to possible existence of a
    UCL Master's dissertation from a few years ago on the subject of a DSL for robotics. I shall
    attempt to find a copy.}

\section{The what and why of functional languages}
To motivate and to clarify the discussion, it is necessary to derive some picture of what might
constitute a functional language, and why it might be a useful way to teach programming to younger
learners. Being more a philosophical stance than technical necessity, it is possible to mix various
paradigms together to form \textit{hybrid} languages. In practice, most programming languages take
precisely this approach\footnote{Truly pure languages are a rare thing: in the realms of OOP perhaps
only \textit{Smalltalk} and \textit{Ruby} qualify. In the functional language family, the only
mainstream entirely pure language is Haskell.}. Python, for example, supports a great deal of the 
functional paradigm, as does JavaScript. These already exist on the micro:bit: why not just use those? 
Consider the following simple scenario, we want to sum all the occurrences of the number 3 in the
following list: 

\mint{python}|nums = [2,3,4,3,2,3,5]|

To do this in an imperative manner, something like the following might be written:

\begin{minted}{python}
    sum = 0
    for i in range(len(nums)):
        if nums[i] == 3:
        sum += 3
\end{minted}

To solve this in Python using functional concepts, something like this might be used:

\mint{python}|sum(list(filter((lambda x: x == 3), nums)))|

This is clearly an improvement in terms of brevity, and is perhaps a little easier to understand. The
main difficulty here in the physical writing of the code is balancing the parentheses, but any good
editor should be able to do this automatically. The filter
higher order function \textit{filter} has a fairly clear meaning to any native English speaker.
However the rather obscure use of \textit{list} and, even worse, \textit{lambda} still makes
this rather difficult for the student. There is a number of topics which would need to be explained
here which reduce the utility of this as a teaching example. Why is the syntax still relatively
cumbersome? Because Python was not conceived as a functional language, many of these features have
been grafted onto a traditional imperative substructure, and the syntax reveals this.
The equivalent function in Haskell would be:

\mint{haskell}|sum (filter (==3) nums)|

Haskell provides some syntactic sugar to reduce the number of brackets even further\footnote{Apologies for the slightly misleading typography
    here, the package being used for syntax highlighting seems unable to properly interpret \LaTeX escape symbols. It should read simply \$.}


\mint{haskell}|sum \$ filter (==3) nums|    
To make all of this code more generic the Python needs to be wrapped in a function definition:

\begin{minted}{python}
    def sumAllOccurrences(n, nums):
        sum = 0
        for i in range(len(nums)):
            if nums[i] == n:
                sum += n
        return sum
\end{minted}

And the equivalent in Haskell:

\mint{haskell}|sumAllOccurrences n ns = sum \$ filter (==n) ns|

In fact, we can do even better than this:

\mint{haskell}|sumAllOccurrences p ns = sum \$ filter p ns|

where p is any predicate we care to pass in, such as (==3) or (\textless2). The python code does not prevent
someone from passing in junk values (leading to a runtime error) but the Haskell code, with the
addition of a simple \textit{type signature} can easily be altered to only receive integers,
ensuring type safety. This need not be explained to the student in these terms, but they will
benefit from their program not crashing in unexpected and difficult to detect ways.

\begin{minted}{haskell}
    sumAllOccurrences :: (Int -> Bool) -> [Int] -> Int
    sumAllOccurrences p ns = sum \$ filter p ns
\end{minted}

This is more powerful than the equivalent Python for loop definition, and a great deal more
flexible. Obviously this could also be implemented in python using filters and lambda expressions,
but it would still need to be couched inside a function definition.\footnote{This is slightly
disingenuous, as a list comprehension, in both Python and Haskell, would work well here. It does
serve to illustrate the general point. Of course, list comprehensions were first introduced in
functional programming languages.}. In this simple example much of the essence of the functional
programming style can be seen. The advantages are clear: fewer lines of code usually means fewer bugs and less
development time. Students can get to solving problems almost immediately, without the worry (or
rather a reduced worry) about the syntactical correctness of their loops, or balancing a host of
nested parentheses. 
Haskell is a large and complicated language and not suitable for younger students. Its runtime is
large and complex, and would almost certainly not run efficiently on the micro:bit. Its emphasis on
purity has led it to introduce a number of features which, while theoretically elegant, are
complex even for experienced programmers. There is however a family of other, related, languages, mostly
descended directly or indirectly from ML\footnote{Excepting the Lisp family representatives,
\textit{Common Lisp} and \textit{Scheme}}, which place different emphasis on parts of the functional
paradigm.\footnote{The table is a simplified (and slight corrected) version of the available from
\url{https://en.wikipedia.org/wiki/Comparison_of_functional_programming_languages}. Monads, monoids
and functors, while important concepts in functional languages, are definitely out of scope for a
teaching language to teenagers. The approach taken by \textit{Miranda} is that which will be followed
in the suggested implementation, hiding unnecessary detail with the loss of a little referential
transparency.}

\begin{center}
\addvbuffer[20pt 20pt]
{\begin{tabular}{|c|cccccc|}
    \hline
    & Pure & Lazy Evalulation & Typing & Algebraic Data Types & Immutable Data & Closures \\
    \hline
    Common Lisp    & No & \cellcolor{LRed}Yes & Dynamic & \cellcolor{GoldenRod}Yes & No & \cellcolor{LRed}Yes \\
    Scheme  & No & No & Dynamic & No & No & \cellcolor{LRed}Yes \\
    ML & No & No & \cellcolor{Lawngreen}Static & \cellcolor{GoldenRod}Yes & No & \cellcolor{LRed}Yes \\
    F\# & No & \cellcolor{LRed}Yes & \cellcolor{Lawngreen}Static & \cellcolor{GoldenRod}Yes & No & \cellcolor{LRed}Yes \\
    Clojure & No & \cellcolor{LRed}Yes & Dynamic & \cellcolor{GoldenRod}Yes & \cellcolor{Seagreen}Yes & \cellcolor{LRed}Yes \\
    Miranda & \cellcolor{Seagreen}Yes & \cellcolor{LRed}Yes & \cellcolor{Lawngreen}Static & \cellcolor{GoldenRod}Yes & \cellcolor{Seagreen}Yes & \cellcolor{LRed}Yes \\
    Haskell & \cellcolor{Seagreen}Yes & \cellcolor{LRed}Yes & \cellcolor{Lawngreen}Static & \cellcolor{GoldenRod}Yes & \cellcolor{Seagreen}Yes & \cellcolor{LRed}Yes \\
    \hline
\end{tabular}}
\end{center}

The table shows clearly that there are a few central features which are associated with functional
languages, and other features which are less essential. Bearing in mind the target audience and the
target device, is should be fairly safe to abandon monads and monoids. Higher order functions are
essential to the paradigm, as is partial function application.\footnote{i.e allowing functions to
    accept fewer arguments than their arity would suggest. The classic example is \mint{haskell}|inc = (+1)| where
    the + operator has been partially applied.}

\section{Requirements \& Constraints: Language Design}
Miranda\textsuperscript{TM} is a language created by David Turner in the early 1980s\footnote{The wikipedia 
    page says that Miranda supports monads. This is not correct, in
    the sense that it is not possible for the programmer to create monad instances or explicitly
    interact with monads. It is also
unlikely that Miranda is using monadic concepts at the implementation level as these were first
introduced into Haskell a number of years after the introduction of Miranda}. It is a member
of the ML family, one of the first purely functional programming languages, and the direct parent of
Haskell and Haskell's various offshoots. Indeed, one of the reasons for the creation of Haskell was
the fact that Miranda was trademarked and closed source. It might best be described as a (small)
subset of Haskell\footnote{Though it would be more accurate to describe Haskell as a superset of
    Miranda.}. It is still used as a teaching language and has a very clean, simply syntax. A
reduced version of Miranda is suggested for the language, which will go under the working title of
microML.\footnote{microMiranda would probably be a more honest name, but there might be a small
    risk of copyright infringement.}

Miranda has an extensive standard library, which comes automatically with the interpreter. MicroML
should have a subset of this standard library, with functions such as map, filter and fold\footnote{Miranda 
    distinguishes between left and right folds. This, while very flexible, might be
too complex for most students. Both Clojure and Python have a simple \textit{reduce} which has the
behaviour of a right fold. In Clojure \textit{apply} is similar to a left fold.}. As far as possible, 
all built-in functions should be \textit{safe}, that is, there should be no
partial functions in the standard library. A partial function is one which does not produce a legal
result for every legal input, for example taking the head of an empty list in Haskell or Miranda
results in a error.

\begin{minted}{haskell}
    head []
    *** Exception: Prelude.head: empty list
\end{minted} 

However this is not a necessity\footnote{The head function in Haskell's standard prelude is a legacy
from Miranda which unfortunately is too late to change easily.} and safe versions of these functions
should be written for microML. 

Laziness is an interesting feature of many functional languages, but it is a little unclear at the
moment what this would mean for programming on the micro:bit. Laziness is chiefly useful when dealing
with infinite data structures, which is an unlikely scenario for the micro:bit.

Another of the more interesting features of the ML family is type inference and static typing. This
is unlikely to be of huge interest to the average student, but it has many benefits. Most
importantly, errors which in a language like Python would only be caught at runtime do not get past
the compilation stage of an ML-style language. Student-friendly (rather than detailed
programmer-friendly) error messages will have to be
written to help the student understand what they have done wrong. Type inference should limit the
student's interaction with the type system, which can be very difficult to understand.

Every functional language has algebraic data types. They are fundamental to how any large program is
constructed. They can be used to simulate enumerations in other languages and as value constructors
(vital in constructing ASTs for example). With regard to the field of robotics for example, to
should be possible to create ADTs which describe the behaviour and attributes of the robot.

Immutability of data is a key (but not universal) feature of the functional paradigm. It allows for
referential transparency and an easier approach to concurrency. Again, these are not features which
are vital to the micro:bit, but they will help students in the future who wish to continue their
coding. Indeed, immutability is in many ways more intuitive for students. If x = 6, then it is 6 for
the entire program. This is a much simpler concept than having a value for x which can go up and down,
or even change type.

\section{Implementation Proposals}
Initial thoughts on the project suggest a number of different ways that the proposed microML might
be implemented. The following is a list of suggestions, with various \textit{pros} and \textit{cons}
listed for each.
The front end of the compiler will be implemented in Haskell. There are a number of excellent
libraries\footnote{Most notably the superb \textit{Parsec} library of parser combinators.} 
and tutorials in this area which should make this part of the compiler easier to
create. The language will also feature a repl for quick exploration and validation of code. It ought to be possible
to link this easily to an on-line editor if the student does not wish, or is unable, to install the
binary. The targeted intermediate representation (henceforth \textit{IR}) will depend on the choice
of back-end.

\begin{enumerate}
    \item Compile directly to C
        \begin{itemize}
                \item Probably the most robust method available. The source code could be translated
                    directly into C, with simple bindings for the micro:bit API.
                \item A major problem would be the necessity to write a garbage collector and
                    runtime, in C, for the language. Writing a industrial strength microML to C compiler and runtime would be a large amount of 
                    work. Time constraints are rather against this.
        \end{itemize}
    \item Compile to a \textit{IR-Language} to C compiler
        \begin{itemize}
            \item The front-end would compile to an IR which could then be passed to an IR to C
                    compiler. If this language already had a runtime environment and garbage
                    collection this approach would take advantage of it. An obvious choice would be a \textit{Scheme}. 
                    There are at least 3 well-maintained scheme to C compilers which are available and open source:
                    Chicken, Gambit-C and Bigloo\footnote{All of these are published under either a
                    BSD license or Apache.}. It would be much easier to translate microML into
                    another language which is also based upon the lambda calculus. This would
                    definitely allow for the creation of a robust product within the time frame, with
                    the only disadvantage being the reliance on another, rather niche, compiler.
                    However, in the event that the compiler became deprecated, another could always
                    be chosen, assuming the IR was a fairly standard scheme\footnote{The
                    \textit{R5RS} standard still seems the most widely supported, so that would make
                    for a sensible target.}.
            \item Compiling to a lisp also has the advantage that it would be easier, due to lisp's
                macro programming, to implement lazy evaluation and pattern matching, both features
                which would be quite complicated to write well in C or microPython.
        \end{itemize}
    \item Compile to microPython or JavaScript
        \begin{itemize}
                \item Perhaps the easiest option is to compile to one of the languages already
                    hosted on the micro:bit. That way the runtime environment and garbage collection
                    of the target could be utilized easily. Moreover, as the micro:bit API is already
                    exposed, it makes it relatively trivial to link this functionality into microML.
                    Python would be a logical choice as the project itself is open source\footnote{MIT license}
                    and implements a reasonably full subset of Python3. Further research would be
                    needed, but it ought to be possible to embed simple, purely functional
                    functionality into the language. Certainly there already exist libraries for
                    lazy evaluation and pattern matching, but it is unlikely that these would run in
                    their present form in microPython. 
                \item Targeting a language already hosted on the micro:bit exposes microML to the
                    danger of the trying to hit a \textit{moving target}. Both microPython and 
                    JavaScript are young implementations and are likely to see many changes made 
                    to their code base and API. This could lead to frequent breaking changes in microML.
        \end{itemize}
\end{enumerate}

\section{Conclusion}
Functional programming, often thought of as the domain only of researchers, is perfectly able to
address real-world problems, and offers a solid basis for students learning to code. A large number
of fundamental computer science concepts can be treated in the context of problem solving, with very
little typing involved (good for the younger students) and very little obfuscation from the
language's syntax. I believe that the approach listed above, of compiling to a lisp-like language and
then taking advantage of a well-maintained compiler, would be the most sensible method, given how
much time is available for coding, testing and debugging.\footnote{The recommended timetable for the
    dissertation gives only between 6 to 8 weeks for coding.}
\end{document}
