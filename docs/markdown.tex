An example comment is taken from the standard library is:

\begin{minted}{sml}
    (*
    ==odd?==
    ***odd?***
    ** odd? :: Number -> Boolean **
    odd? checks if a number is odd, 
     returning a __boolean__ value.
    \#Example:\#
        > odd? 5
        > true : Boolean
        > odd? 4
        > false : Boolean
    *)
\end{minted}

\begin{itemize}
    \item a \textbf{comment} begins with `(*' and ends with `*)'
    \item the \textbf{comment name} is surrounded with `=='. This is the lookup string for the repl
        search.
    \item the \textbf{repl title} is surrounded with `***'. This text will be highlighted by a yellow
        bar in the repl.
    \item text to be rendered \textbf{bold} is surrounded with `**'
    \item text to be \underline{underlined} is surrounded with `__'
    \item headers are surrounded with `\#' 
    \item plain text is left without markup.
\end{itemize}

While not enforced by the syntax itself, underlined words are those which can be found in the
glossary. The glossary is a list of words, written with this markup, which might be unfamiliar to
students.
